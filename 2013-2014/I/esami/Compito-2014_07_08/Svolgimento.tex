\documentclass[a4paper]{report}

\usepackage[top=1cm, bottom=1cm]{geometry}
\usepackage{amsmath}    % {aligned}
\usepackage{amssymb}    % \mathbb
\usepackage{mathtools}  % {dcases}
\usepackage{relsize}    % \mathlarger
\usepackage{soul}       % \ul
\usepackage{tikz}       % \tikz, \node
\usepackage{mdframed}   % {mdframed}

\newenvironment{problem}
        {
                \begin{mdframed}[topline=false,rightline=false,bottomline=false]
                        \slshape
        }
        {
                \end{mdframed}
        }

\newcommand*\circled[1]{
        \tikz[baseline=(char.base)]{
                \node[shape=circle,draw,inner sep=1pt] (char) {#1};
        }
}


\begin{document}
        %%%%%%%%%%%%%%%%%%%%%%%
        \chapter*{Esercizio A1}
        %%%%%%%%%%%%%%%%%%%%%%%
        \begin{problem}
                Determinare se i tre vettori di $\mathbb{R}^3: (1,2,-2), (-1,3,3) \mbox{ e } (1,3,-2)$
                siano o no linearmente indipendenti.
        \end{problem}

        \subsection*{Metodo 1}
        Se $det(M_{atrice}) \neq 0$ allora i vettori sono linearmente indipendenti.
        \[
                M =
                \begin{pmatrix}
                        1  & -1 &  1 \\
                        2  &  3 &  3 \\
                        -2 &  3 & -2
                \end{pmatrix};
        \]

        $
                det(M) = (1 \cdot 3 \cdot -2) + (-1 \cdot 3 \cdot -2) + (1 \cdot 2 \cdot 3) - (1 \cdot 3 \cdot -2) - (-1 \cdot 2 \cdot -2) - (1 \cdot 3 \cdot 3) = -6 + 6 + 6 + 6 -4  -1.
        $

        \paragraph{}
        Il determinante \`{e} diverso da zero, quindi possiamo concludere che \ul{i vettori sono linearmente indipendenti}.

        \subsection*{Metodo 2}
        Se $ a v_1 + b v_2 + c v_3 = 0 $ \`{e} risolto solo per $(a, b, c) = (0, 0, 0)$
        allora i vettori sono linearmente indipendenti.

        \paragraph{}
        $
                a \begin{pmatrix} 1 \\ 2 \\ -2 \end{pmatrix} + b \begin{pmatrix} -1 \\ 3 \\ 3 \end{pmatrix} + c \begin{pmatrix} 1 \\ 3 \\ -2 \end{pmatrix} = \begin{pmatrix} 0 \\ 0 \\ 0 \end{pmatrix}
        $

        \paragraph{}
        $
                \begin{dcases}
                        a   - b  + c  & = 0 \\
                        2a  + 3b + 3c & = 0 \\
                        -2a + 3b - 2c & = 0
                \end{dcases}; \quad \begin{aligned}
                        a & = b - c \\
                        2a  + 3b + 3c & = 0 \\
                        b & = 0
                \end{aligned}; \quad \begin{aligned}
                        a & = 0 \\
                        c & = 0 \\
                        b & = 0
                \end{aligned} \quad .
        $

        \paragraph{}
        $(0,0,0)$ \`{e} l'unica soluzione del sistema, quindi possiamo concludere
        che \ul{i vettori sono linearmente indipendenti}.

        \subsection*{}
        \begin{problem}
                Determinare la dimensione del sottospazio da essi generato.
        \end{problem}

        \paragraph{}
        I tre vettori costituiscono una base, essendo linearmente indipendenti
        ed essendo un sistema di generatori (avendo rango massimo, ovvero uguale a 3).
        Concludiamo che \ul{$dim(\{v_1, v_2, v_3\}) = 3$, ovvero tutto $\mathbb{R}^3$}.


        %%%%%%%%%%%%%%%%%%%%%%%
        \chapter*{Esercizio A2}
        %%%%%%%%%%%%%%%%%%%%%%%
        \begin{problem}
                Determinare il nucleo della trasformazione lineare da $\mathbb{R}^3$ in $\mathbb{R}^3$ rappresentata, rispetto alla base canonica, dalla seguente matrice:
                \[
                        \begin{pmatrix}
                               1 & 2 & 4 \\
                               0 & 1 & 2 \\
                               1 & 0 & 3
                        \end{pmatrix}.
                \]
        \end{problem}

        \paragraph{}
        Il nucleo \`{e} l'insieme di vettori soluzione del sistema di equazioni omogeneo $Ax = 0$.
        Quindi per trovarlo \`{e} sufficiente trovare le soluzioni del sistema:
        \[
                \begin{dcases}
                        x + 2y + 4z & = 0 \\
                        y + 2z      & = 0 \\
                        x + 3z      & = 0
                \end{dcases}
                \begin{aligned}
                        x  & = 0 \\
                        y  & = 0 \\
                        z  & = 0
                \end{aligned}.
        \]

        \framebox{$Ker(F) = \{\}$ e $dim(Ker(F)) = 0$}.


        %%%%%%%%%%%%%%%%%%%%%%%
        \chapter*{Esercizio A3}
        %%%%%%%%%%%%%%%%%%%%%%%
        \begin{problem}
                Determinare gli autovalori e gli autovettori della seguente matrice:
                $
                        \begin{pmatrix}
                                4 & 2 & 0 \\
                                1 & 3 & -1 \\
                                0 & 0 & -3
                        \end{pmatrix}.
                $
        \end{problem}

        \paragraph{}
        Sfruttiamo la definizione di autovalore ed autovettore:
        $A_{f} - \lambda I$ deve essere non invertibile,
        ovvero il $det(A_{f} - \lambda I) = 0$.

        \[
                A_{f} -\lambda I = \begin{pmatrix}
                        4 & 2 & 0 \\
                        1 & 3 & -1 \\
                        0 & 0 & -3
                \end{pmatrix} - \begin{pmatrix}
                        \lambda & 0       & 0 \\
                        0       & \lambda & 0 \\
                        0       & 0       & \lambda
                \end{pmatrix} = \begin{pmatrix}
                        4 - \lambda & 2 & 0 \\
                        1 & 3 - \lambda & -1 \\
                        0 & 0 & -3 - \lambda
                \end{pmatrix};
        \]

        \[
                det(A_{f} - \lambda I) = (- 3 - \lambda) [ (4 - \lambda) (3 - \lambda) - 2) ] = (- 3 - \lambda) (10 - 7 \lambda + \lambda^{2}) = 0 ;
        \]

        \paragraph{}
        $
                \lambda_{1} = - 3;
        $

        \paragraph{}
        $
                \lambda_{2,3} = \mathlarger{\frac{7 \pm \sqrt{49 - 10 }}{2}} = \mathlarger{\frac{7 \pm 6}{2}} = 13/2, 1/2;
        $

        \paragraph{}
        Per l'autovalore \framebox{$\lambda_1 = -3$}:

        \[
                \begin{pmatrix}
                        7 & 2 & 0 \\
                        1 & 6 & -1 \\
                        0 & 0 & 0
                \end{pmatrix} \begin{pmatrix}
                        x \\
                        y \\
                        z
                \end{pmatrix} = \begin{pmatrix}
                        0 \\
                        0 \\
                        0
                \end{pmatrix} \implies
                \begin{dcases}
                        7x + 2y & = 0 \\
                        x + 6y - z & = 0 \\
                        0 = 0
                \end{dcases}\;; \quad \begin{aligned}
                        7w + 2y & = 0 \\
                        w + 6y - z & = 0 \\
                        x = w
                \end{aligned}\;; \quad \begin{aligned}
                        y & = -7/2w \\
                        z & = -20w \\
                        x & = w
                \end{aligned} \; ;
        \]

        l'autovetore \`{e} $(w, -7/2w, -20w)$, ovvero \framebox{$w(1, -7/2, -20)$}.

        \paragraph{}
        Per gli autovalori $\lambda_2$ e $\lambda_3$ si applica lo stesso procedimento.


        %%%%%%%%%%%%%%%%%%%%%%%
        \chapter*{Esercizio B1}
        %%%%%%%%%%%%%%%%%%%%%%%
        \begin{problem}
                Determinare gli interi $a,b$ tali che $ 10a + 23b = 1 $.
        \end{problem}

        \paragraph{}
        Sfruttiamo il teorema di B\'{e}zout.\\
        \emph{
                Se $a,b$ sono interi e $(a,b)$ \`{e} il loro massimo comune divisore,
                allora esistono interi $h,k$ tali che
                $ ha + kb = (a, b) $.
        }

        \paragraph{$MCD(10,23)$:}
        {
                \setlength{\tabcolsep}{2pt}
                \begin{tabular}{r l l l}
                        23 = & 10 & * 2 + \circled{1}; \\
                        10 = & 2  & * 5 + \ul{0}.
                \end{tabular}
        }

        \paragraph{}
        Il massimo comune divisore di $10$ e $23$ \`{e} proprio $1$,
        quindi il teorema di B\'{e}zout ci garantisce che esistono due interi che soddisfano l'equazione.

        \noindent
        Per ricavarli \`{e} sufficiente risalire ricorsivamente (in questo caso si tratta di un solo passaggio) l'algoritmo di Euclide tenendo a mente che
        $ a = qb + r \implies r = a - qb $

        \paragraph{}
        Quindi $1 = 23 - 10 \cdot 2$, con \framebox{$ a = -2, b = 1 $}.


        %%%%%%%%%%%%%%%%%%%%%%%
        \chapter*{Esercizio B2}
        %%%%%%%%%%%%%%%%%%%%%%%
        \begin{problem}
                Quali sono gli argomenti dei numeri complessi $z$ di modulo $3$ tali che $Re(z) + Im(z) = 0$?
        \end{problem}

        \paragraph{}
        Affinch\'{e} sia $Re(z) + Im(z) = 0$ deve valere che $Re(z) = - Im(z)$, ovvero $x = -iy$, quindi $iy = -x$.

        \paragraph{}
        $
                |z| = \sqrt{{Re(z)}^2 + {Im(z)}^2} = \sqrt{x^2 + x^2} = x \sqrt{2} = 3 \quad \implies \quad
                x = \frac{3}{\sqrt{2}};
        $

        \paragraph{}
        $
                \begin{aligned}
                        x & = |z| \cos{\theta} \implies
                                \frac{3}{\sqrt{2}} = 3 \cos{\theta} \implies
                                \cos{\theta} = \frac{1}{\sqrt{2}} \\
                        y & = |z| \sin{\theta} \implies
                                -\frac{3}{\sqrt{2}} = 3 \sin{\theta} \implies
                                \sin{\theta} = -\frac{1}{\sqrt{2}}
                \end{aligned};
                \quad \implies \quad
        $
        \framebox{$\theta = \frac{\pi}{4} + k 2 \pi$}.


        %%%%%%%%%%%%%%%%%%%%%%%
        \chapter*{Esercizio B3}
        %%%%%%%%%%%%%%%%%%%%%%%
        \begin{problem}
                Determinare la parit\`{a} della permutazione (247536)(45)(1745932) in $S_{11}$.
        \end{problem}

        \paragraph{}
        $
                \begin{aligned}
                        1 & \to 7 \to 5; \\
                        5 & \to 9; \\
                        9 & \to 3 \to 6; \\
                        6 & \to 2; \\
                        2 & \to 1. \\
                        \hline
                        3 & \to 2 \to 4; \\
                        4 & \to 5 \to 4 \to 7; \\
                        7 & \to 4 \to 5 \to 3. \\
                        \hline
                \end{aligned}
        $

        \paragraph{}
        $ (15962)(347) $

        \paragraph{}
        $ (5 - 1) + (3 - 1) = 6 $ \framebox{Pari (+1)}.

        \paragraph{Extra:}
        $ (12)(16)(19)(15)(37)(34) $.

\end{document}
