\documentclass[a4paper]{report}

\usepackage[top=1cm, bottom=1cm]{geometry}
\usepackage{amsmath}    % {aligned}
\usepackage{amssymb}    % \mathbb
\usepackage{mathtools}  % {dcases}
\usepackage{relsize}    % \mathlarger
\usepackage{soul}       % \ul
\usepackage{tikz}       % \tikz, \node
\usepackage{mdframed}   % {mdframed}

\newenvironment{problem}
        {
                \begin{mdframed}[topline=false,rightline=false,bottomline=false]
                        \slshape
        }
        {
                \end{mdframed}
        }

\newcommand*\circled[1]{
        \tikz[baseline=(char.base)]{
                \node[shape=circle,draw,inner sep=1pt] (char) {#1};
        }
}


\begin{document}
        %%%%%%%%%%%%%%%%%%%%%%%
        \chapter*{Esercizio A1}
        %%%%%%%%%%%%%%%%%%%%%%%
        \begin{problem}
                Determinare se i tre vettori di $\mathbb{R}^3: (1,2,0), (0,8,3) \mbox{ e } (1,0,-2)$
                siano o no linearmente indipendenti.
        \end{problem}

        \subsection*{Metodo 1}
        Se $det(M_{atrice}) \neq 0$ allora i vettori sono linearmente indipendenti.
        \[
                M =
                \begin{pmatrix}
                        1 & 0 & 1 \\
                        2 & 8 & 0 \\
                        0 & 3 & -2
                \end{pmatrix};
        \]

        $
                det(M) = (1 \cdot 8 \cdot -2) + (0 \cdot 0 \cdot 0) + (2 \cdot 3 \cdot 1) - (1 \cdot 8 \cdot 0) - (0 \cdot 2 \cdot -2) - (1 \cdot 0 \cdot 3) = -10.
        $

        \paragraph{}
        Il determinante \`{e} diverso da zero, quindi possiamo concludere che \ul{i vettori sono linearmente indipendenti}.

        \subsection*{Metodo 2}
        Se
        $
                a v_1 + b v_2 + c v_3 = 0
        $
        \`{e} risolto solo per $(a, b, c) = (0, 0, 0)$ allora i vettori sono linearmente indipendenti.

        \paragraph{}
        $
                a \begin{pmatrix} 1 \\ 2 \\ 0 \end{pmatrix} + b \begin{pmatrix} 0 \\ 8 \\ 3 \end{pmatrix} + c \begin{pmatrix} 1 \\ 0 \\ -2 \end{pmatrix} = \begin{pmatrix} 0 \\ 0 \\ 0 \end{pmatrix}
        $

        \paragraph{}
        $
                \begin{dcases}
                        a + c   & = 0 \\
                        2a + 8b & = 0 \\
                        3b - 2c & = 0
                \end{dcases}; \quad \begin{aligned}
                        a & = -c \\
                        c & = 4b \\
                        b & = 0
                \end{aligned}; \quad \begin{aligned}
                        a & = 0 \\
                        b & = 0 \\
                        c & = 0
                \end{aligned} \quad .
        $

        \paragraph{}
        $(0,0,0)$ \`{e} l'unica soluzione del sistema, quindi possiamo concludere
        che \ul{i vettori sono linearmente indipendenti}.


        \subsection*{}
        \begin{problem}
                Determinare la dimensione del sottospazio da essi generato.
        \end{problem}

        \paragraph{}
        I tre vettori costituiscono una base, essendo linearmente indipendenti
        ed essendo un sistema di generatori (avendo rango massimo, ovvero uguale a 3).
        Concludiamo che \ul{$dim(\{v_1, v_2, v_3\}) = 3$, ovvero tutto $\mathbb{R}^3$}.


        %%%%%%%%%%%%%%%%%%%%%%%
        \chapter*{Esercizio A2}
        %%%%%%%%%%%%%%%%%%%%%%%
        \begin{problem}
                Ridurre a scala la matrice:\quad
                $
                        \begin{matrix}r_1 \\ r_2 \\ r_3\end{matrix}
                        \begin{pmatrix}
                               1 & 2 & 3 & 2 &  0 \\
                               0 & 1 & 0 & 2 & -3 \\
                               1 & 0 & 3 & 0 & 1
                        \end{pmatrix}
                $.
        \end{problem}

        \paragraph{}
        $r_i = r_i - \mathlarger{\frac{a_i}{a_p}} r_p$.

        \paragraph{}
        $
                r_1 = r_1 - 1 r_2 = \begin{pmatrix}1 & 0 & 3 & 0 & 1\end{pmatrix} - 1\begin{pmatrix}1 & 2 & 3 & 2 & 0\end{pmatrix}
                \quad \implies \quad
                \begin{matrix}r_1 \\ r_2 \\ r_3\end{matrix}
                \begin{pmatrix}
                        1 & 2  & 3 & 2  &  0 \\
                        0 & 1  & 0 & 2  & -3 \\
                        0 & -2 & 0 & -2 & 1
                \end{pmatrix};
        $

        \paragraph{}
        $
                r_3 = r_3 + 2 r_2 = \begin{pmatrix}0 & -2 & 0 & -2 & 1\end{pmatrix} + 2 \begin{pmatrix}0 & 1 & 0 & 2 & -3\end{pmatrix}
                \quad \implies \quad
                \begin{matrix}r_1 \\ r_2 \\ r_3\end{matrix}
                \begin{pmatrix}
                        1 & 2 & 3 & 2  &  0 \\
                        0 & 1 & 0 & 2  & -3 \\
                        0 & 0 & 0 & 2  & -5
                \end{pmatrix};
        $

        \`{E} una \emph{matrice triangolare superiore}.

        \paragraph{}
        $
                r_1 = r_1 - 2 r_2 = \begin{pmatrix}1 & 2 & 3 & 2 & 0\end{pmatrix} - 2 \begin{pmatrix}0 & 1 & 0 & 2 & -3\end{pmatrix}
                \quad \implies \quad
                \begin{matrix}r_1 \\ r_2 \\ r_3\end{matrix}
                \begin{pmatrix}
                        1 & 0 & 3 & 2  &  6 \\
                        0 & 1 & 0 & 2  & -3 \\
                        0 & 0 & 0 & 2  & -5
                \end{pmatrix};
        $

        \paragraph{}
        $
                r_2 = r_2 - 1 r_3 = \begin{pmatrix}0 & 1 & 0 & 2 & -3\end{pmatrix} - 1 \begin{pmatrix}0 & 0 & 0 & 2 & -5\end{pmatrix}
                \quad \implies \quad
                \begin{matrix}r_1 \\ r_2 \\ r_3\end{matrix}
                \begin{pmatrix}
                        1 & 0 & 3 & 2  &  6 \\
                        0 & 1 & 0 & 0  &  2 \\
                        0 & 0 & 0 & 2  & -5
                \end{pmatrix};
        $

        \paragraph{}
        $
                r_1 = r_1 - 1 r_3 = \begin{pmatrix}1 & 0 & 3 & 2 & 6\end{pmatrix} - 1 \begin{pmatrix}0 & 0 & 0 & 2 & -5\end{pmatrix}
                \quad \implies \quad
                \begin{matrix}r_1 \\ r_2 \\ r_3\end{matrix}
                \begin{pmatrix}
                        1 & 0 & 3 & 0  & 11 \\
                        0 & 1 & 0 & 0  &  2 \\
                        0 & 0 & 0 & 2  & -5
                \end{pmatrix};
        $

        \`{E} una \emph{matrice triangolare superiore ed inferiore}.

        \paragraph{}
        Divido le righe con $pivot \neq 1$ per il $pivot$, per ottenere pivot tutti uguali a $1$.
        \[
                \begin{pmatrix}
                        \circled{1} & 0 & 3 & 0  & 11 \\
                        0 & \circled{1} & 0 & 0  &  2 \\
                        0 & 0 & 0 & \circled{1}  & -5/2
                \end{pmatrix}.
        \]

        \`{E} una \emph{matrice triangolare}.


        %%%%%%%%%%%%%%%%%%%%%%%
        \chapter*{Esercizio A3}
        %%%%%%%%%%%%%%%%%%%%%%%
        \begin{problem}
                Determinare gli autovalori e gli autovettori della trasformazione lineare
                di $\mathbb{R}^2$ in $\mathbb{R}^2$ che manda i due vettori $e_1$, $e_2$
                della base canonica di $\mathbb{R}^2$ rispettivamente in $(0, 8)$ e $(1, 2)$.
        \end{problem}

        \paragraph{}
        Sfruttiamo la definizione di autovalore ed autovettore:
        $A_{f} - \lambda I$ deve essere non invertibile,
        ovvero il $det(A_{f} - \lambda I) = 0$.

        \[
                A_{f} = \begin{pmatrix}
                        v_1 & v_2
                \end{pmatrix} = \begin{pmatrix}
                        0 & 1 \\
                        8 & 2
                \end{pmatrix};
        \]

        \[
                A_{f} -\lambda I = \begin{pmatrix}
                        0 & 1 \\
                        8 & 2
                \end{pmatrix} - \begin{pmatrix}
                        \lambda & 0 \\
                        0       & \lambda
                \end{pmatrix} = \begin{pmatrix}
                        -\lambda & 1 \\
                        8        & 2 - \lambda
                \end{pmatrix};
        \]

        \[
                det(A_{f} - \lambda I) = - \lambda (2 - \lambda) - 8 = \lambda^{2} - 2 \lambda - 8;
        \]

        \[
                \lambda_{1,2} = \frac{2 \pm \sqrt{4 + 32}}{2} = \frac{2 \pm 6}{2} = 4, -2;
        \]

        \paragraph{}
        Per l'autovalore $\lambda_1 = 4$:

        \[
                \begin{pmatrix}
                        -4 & 1 \\
                        8  & -2
                \end{pmatrix} \begin{pmatrix}
                        x \\
                        y
                \end{pmatrix} = \begin{pmatrix}
                        0 \\
                        0
                \end{pmatrix} = \begin{dcases}
                        -4x + y & = 0 \\
                        8x - 2y & = 0
                \end{dcases}\;; \quad \begin{aligned}
                        y & = 4x \\
                        8x - 8x & = 0
                \end{aligned}\;; \quad \begin{aligned}
                        y & = 4x \\
                        0 & = 0
                \end{aligned}; \quad \begin{aligned}
                        y & = 4w \\
                        x & = w
                \end{aligned}\;;
        \]

        l'autovetore \`{e} $(w, 4w)$, ovvero $w(1, 4)$ .


        \paragraph{}
        Per l'autovalore $\lambda_2 = - 2$:

        \[
                \begin{pmatrix}
                        2 & 1 \\
                        8  & 4
                \end{pmatrix} \begin{pmatrix}
                        x \\
                        y
                \end{pmatrix} = \begin{pmatrix}
                        0 \\
                        0
                \end{pmatrix} = \begin{dcases}
                        2x + y  & = 0 \\
                        8x + 4y & = 0
                \end{dcases}\;; \quad \begin{aligned}
                        y & = -2x \\
                        8x - 8x & = 0
                \end{aligned}\;; \quad \begin{aligned}
                        y & = -2x \\
                        0 & = 0
                \end{aligned}; \quad \begin{aligned}
                        y & = -2w \\
                        x & = w
                \end{aligned}\;;
        \]

        l'autovetore \`{e} $(w, -2w)$, ovvero $w(1, -2)$ .


        %%%%%%%%%%%%%%%%%%%%%%%
        \chapter*{Esercizio B1}
        %%%%%%%%%%%%%%%%%%%%%%%
        \begin{problem}
                Determinare con l'algoritmo di Euclide il minimo comune multiplo tra 1988 e 1805.
        \end{problem}

        \paragraph{$MCD(1988, 1805)$:} ~\newline
        {
                \setlength{\tabcolsep}{1pt}
                \begin{tabular}{r l l l}
                        1988 = & 1805 & * 1 + 183 \\
                        1805 = & 183  & * 9 + 158 \\
                        183  = & 158  & * 1 + 25  \\
                        158  = & 25   & * 6 + 8   \\
                        25   = & 8    & * 3 + \circled{1}   \\
                        8    = & 1    & * 8 + \ul{0}
                \end{tabular}
        }

        \paragraph{}
        L'$MCD$ \`{e} l'ultimo resto prima dello $0$, quindi $MCD(1988, 1805) = 1$.

        \paragraph{}
        $mcm(1988, 1805) = \mathlarger{\frac{1988 \cdot 1805}{1}} = \underline{3588340}$.

\end{document}
