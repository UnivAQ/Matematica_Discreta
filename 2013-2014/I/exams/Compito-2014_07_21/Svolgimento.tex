\documentclass[a4paper]{report}


\usepackage[top=1cm, bottom=1cm]{geometry}
\usepackage{amsmath}    % {aligned}
\usepackage{amssymb}    % \mathbb
\usepackage{mathtools}  % {dcases}
\usepackage{relsize}    % \mathlarger
\usepackage{soul}       % \ul
\usepackage{tikz}       % \tikz, \node
\usepackage{mdframed}   % {mdframed}


\newenvironment{problem}
        {
                \begin{mdframed}[topline=false,rightline=false,bottomline=false]
                        \slshape
        }
        {
                \end{mdframed}
        }

\newcommand*\circled[1]{
        \tikz[baseline=(char.base)]{
                \node[shape=circle,draw,inner sep=1pt] (char) {#1};
        }
}


\begin{document}
        %%%%%%%%%%%%%%%%%%%%%%%
        \chapter*{Esercizio A1}
        %%%%%%%%%%%%%%%%%%%%%%%
        \begin{problem}
                Determinare se i vettori $ (1,0,2,-1),(3,3,1,2),(-3,1,0,4),(4,-1,2,-5) $ sono o no una base di $\mathbb{R}^4$.
        \end{problem}

        \paragraph{}
        Affinch\'{e} i vettori siano una base devono essere un sistema di generatori e devono essere linearmente indipendenti.

        \paragraph{}
        Sia
        \[
                M = \begin{pmatrix}
                        1 & 3 & -3 & 4 \\
                        0 & 3 & 1 & -1 \\
                        2 & 1 & 0 & 2  \\
                        -1  & 2 & 4 & -5
                \end{pmatrix}
        \]
        la matrice dei vettori disposti per colonna,
        verifichiamo se sono linearmente indipendenti controllando se il $ det(M) \not= 0 $.

        \[
                det(M) = -46
        \]
        quindi i vettori sono linearmente indipendenti.
        Da ci\`{o} ne deriva che il rango della matrice \`{e} massimo, ovvero 4,
        e quindi sono un sistema di generatori.

        Di conseguenza \ul{sono una base di $ dim(M) = 4 $ e quindi generano tutto $\mathbb{R}^4$}.


        %%%%%%%%%%%%%%%%%%%%%%%
        \chapter*{Esercizio A2}
        %%%%%%%%%%%%%%%%%%%%%%%
        \begin{problem}
                Determinare il rango della trasformazione lineare da $\mathbb{R}^3$ in $\mathbb{R}^3$ rappresentata, rispetto alla base canonica, dalla seguente matrice:
                \[
                        \begin{pmatrix}
                               4 & 2 & 11 \\
                               0 & 1 & 3 \\
                               -1 & 0 & 3
                        \end{pmatrix}.
                \]
        \end{problem}

        \paragraph{}
        La matrice \`{e} gi\`{a} quadrata, quindi si calcola il determinante della matrice $ 3 \times 3 $.

        \[
                det(M) = 17.
        \]

        Il $ det(M) \not= 0$ e quindi \framebox{$ Rk(M) = 3 $}.
        Poich\'{e}' $ Rk(M) = 3 = min(3,3) $, \ul{il rango di $M$ \`{e} massimo}.


        %%%%%%%%%%%%%%%%%%%%%%%
        \chapter*{Esercizio A3}
        %%%%%%%%%%%%%%%%%%%%%%%
        \begin{problem}
                Determinare gli autovalori e gli autovettori dell'operatore lineare di $\mathbb{R}^2$ in $\mathbb{R}^2$
                che manda i due vettori $ e_1, e_2 $ della base canonica di $\mathbb{R}^2$ rispettivamente in
                $(1,1)$ e $(0,5)$.
        \end{problem}

        \paragraph{}
        La matrice canonica dell'applicazione lineare \`{e}
        $
                A =     \bigl(f(e_1),f(e_2)\bigr) =
                        \begin{pmatrix}
                                1 & 0 \\
                                1 & 5
                        \end{pmatrix}.
        $

        \paragraph{}
        Per determinare gli autovalori e gli autovettori si ricorre ad una delle definizioni di autovettore:
        \begin{center}
                $ A - \lambda I $ non \`{e} invertibile, ovvero $ det(A - \lambda I) = 0$.
        \end{center}

        \paragraph{}
        Quindi
        \[
                A - \lambda I = \begin{pmatrix}
                        1 & 0 \\
                        1 & 5
                \end{pmatrix} - \begin{pmatrix}
                        \lambda & 0 \\
                        0 & \lambda
                \end{pmatrix} = \begin{pmatrix}
                        1 - \lambda & 0 \\
                        1 & 5 - \lambda
                \end{pmatrix} \quad \implies
        \]
        \[
                det(A - \lambda I) = \begin{pmatrix}
                        1 - \lambda & 0 \\
                        1 & 5 - \lambda
                \end{pmatrix} = 0 \quad \implies
        \]
        \[
                (1 - \lambda) (5 - \lambda) = 0 \quad \implies \quad \lambda_1 = 5,\ \lambda_2 = 1.
        \]

        \paragraph{}
        Per l'autovalore \framebox{$ \lambda_2 = 1 $},
        $
                \begin{pmatrix}
                        1 - \lambda & 0 \\
                        1 & 5 - \lambda
                \end{pmatrix} = \begin{pmatrix}
                        0 & 0 \\
                        1 & 4
                \end{pmatrix}
        $:
        \[
                \begin{pmatrix}
                        0 & 0 \\
                        1 & 4
                \end{pmatrix}\begin{pmatrix}
                        x \\
                        y
                \end{pmatrix} = \begin{pmatrix}
                        0 \\
                        0
                \end{pmatrix} \quad \implies \quad
                \begin{dcases}
                        0  = 0 \\
                        x + 4y = 0
                \end{dcases} \quad \implies \quad
                \begin{dcases}
                        x  = a \\
                        y = - \frac{a}{4}
                \end{dcases}.
        \]
        l'autovettore \`{e} \framebox{$ a (1, - \frac{1}{4}$)}.

        \paragraph{}
        Per l'autovalore $ \lambda_2 = 5 $,
        $
                \begin{pmatrix}
                        1 - \lambda & 0 \\
                        1 & 5 - \lambda
                \end{pmatrix} = \begin{pmatrix}
                        -4 & 0 \\
                        1 & 0
                \end{pmatrix}
        $:
        \[
                \begin{pmatrix}
                        -4 & 0 \\
                        1 & 0
                \end{pmatrix}\begin{pmatrix}
                        x \\
                        y
                \end{pmatrix} = \begin{pmatrix}
                        0 \\
                        0
                \end{pmatrix} \quad \implies \quad
                \begin{dcases}
                        x  = 0 \\
                        y = 0
                \end{dcases}.
        \]
        l'autovettore non esiste.


        %%%%%%%%%%%%%%%%%%%%%%%
        \chapter*{Esercizio B1}
        %%%%%%%%%%%%%%%%%%%%%%%
        \begin{problem}
                Sia $G$ un gruppo, $ g \in G $, e sia $ \alpha\!:\ \mathbb{Z} \to G $ la funzione
                definita da $ \alpha(n) = x^n $. Mostrare che $\alpha$ \`{e} un omomorfismo di gruppi.
        \end{problem}

        \paragraph{}
        Sia il gruppo $Z$ definito con l'operazione di somma ed il gruppo $G$ con l'operazione di moltiplicazione,
        ovvero $(Z,+), (G,\cdot)$.

        \paragraph{}
        Affinch\'{e} $\alpha$ sia un omomorfismo di gruppi deve valere che
        \[
                \alpha(a + b) = \alpha(a) \cdot \alpha(b) \quad \forall a,b \in Z.
        \]

        \paragraph{}
        Da cui
        \[
                        \alpha(a + b) = x^{a + b} = x^a \cdot x^b = \alpha(a) \cdot \alpha(b).
        \]

        \paragraph{}
        La propriet\`{a} \`{e} verificata, quindi \ul{$\alpha$ \`{e} un omomorfismo fra gruppi}.


        %%%%%%%%%%%%%%%%%%%%%%%
        \chapter*{Esercizio B2}
        %%%%%%%%%%%%%%%%%%%%%%%
        \begin{problem}
                Usare il teorema di Lagrange per mostrare esplicitamente che un gruppo di $12$ elementi
                non ha sottogruppi di ordine $5$ (non basta ricordare l'enunciato del teorema, occorre
                ricordare la dimostrazione).
        \end{problem}

        \paragraph{}
        Dal teorema di Lagrange, con $ |G:H| = r $,
        \[
                |G| =  r \cdot |H|
        \]
        da cui
        \[
                12 = r \cdot 5
        \]
        ovvero non esiste un $r$ che moltiplicato $5$ dia come valore $12$, quindi \ul{un gruppo di $12$ elementi
        non ha sottogruppi di ordine $5$}.


        %%%%%%%%%%%%%%%%%%%%%%%
        \chapter*{Esercizio B3}
        %%%%%%%%%%%%%%%%%%%%%%%
        \begin{problem}
                Risolvere l'equazione $ z = - \overline{z} $ in campo complesso.
        \end{problem}

        \paragraph{}
        Con $ z = x + iy $, il suo coniugato \`{e} definito come $ \overline{z} = x - iy $.

        \paragraph{}
        Da cui
        \[
                x + iy = - (x + iy) \implies x + iy = -x + iy \implies 2x = 0 \implies \framebox{$x = 0$}.
        \]

\end{document}
