\documentclass[a4paper]{article}


\usepackage[top=1cm, bottom=1cm]{geometry}
\usepackage{amsmath}    % {aligned}
\usepackage{amssymb}    % \mathbb
\usepackage{mathtools}  % {dcases}
\usepackage{relsize}    % \mathlarger
\usepackage{setspace}   % {spacing}
\usepackage{soul}       % \ul
\usepackage{tikz}       % \tikz, \node
\usepackage{cancel}     % \cancel


\newcommand*\circled[1]{
        \tikz[baseline=(char.base)]{
                \node[shape=circle,draw,inner sep=1pt] (char) {#1};
        }
}


\begin{document}
        %%%%%%%%%%%%%%%%%%%%%%
        \section*{Esercizio 1}
        %%%%%%%%%%%%%%%%%%%%%%
        \textsl{Nel piano euclideo $E^3$ determinare:}

        \noindent
        \textsl{
                a) l'equazione del piano per $P(1,2,1)$ e parallelo alle rette:
                $
                        r: \begin{dcases} x = z - 1 \\ y = 2z + 3 \end{dcases}; \;
                        s: \begin{dcases} x = -z + 1 \\ y = 3z - 2 \end{dcases}
                $.
        }

        L'equazione cartesiana di un piano passante per un punto $P(x_0, y_0, z_0)$ e parallelo a due rette \`{e} data da
        \[
                det(
                        \begin{pmatrix}
                                x - x_0 & y - y_0 & z - z_0 \\
                                a_1 & b_1 & c_1 \\
                                a_2 & b_2 & c_2 \\
                        \end{pmatrix}
                ) = 0
        \]
        dove $v_1 = (a_1, b_1, c_1)$ e $v_2 = (a_2, b_2, c_2)$ sono rispettivamente i vettori direzione delle due rette.
        Per ricavarli \`{e} sufficiente riscrivere le equazioni delle rette dalla forma cartesiana a quella parametrica.

        \[
                r: \begin{dcases}
                        x = & z - 1 \\
                        y = & 2z + 3
                \end{dcases} \quad \implies \quad
                \begin{dcases}
                        x = & -1 + t \\
                        y = & 3 + 2t \\
                        z = & t
                \end{dcases} \quad \implies \quad
                v_1 = (1, 2, 1).
        \]

        \[
                s: \begin{dcases}
                        x = & -z + 1 \\
                        y = & 3z - 2
                \end{dcases} \quad \implies \quad
                \begin{dcases}
                        x = & 1 - t \\
                        y = & -2 + 3t\\
                        z = & t
                \end{dcases} \quad \implies \quad
                v_2 = (-1, 3, 1).
        \]

        Da cui,
        \[
                \begin{aligned}
                        det(
                                \begin{pmatrix}
                                        x - 1 & y - 2 & z - 1 \\
                                        1     & 2     & 1 \\
                                        -1    & 3     & 1 \\
                                \end{pmatrix}
                        ) & = \framebox{ -x - 2y + 5z = 0 }
                \end{aligned}
        \]

        \noindent
        \textsl{
                b) l'equazione del piano per $P(1,1,1)$ e perpendicolare alla retta:
                $
                        t: \begin{dcases}
                                x + 2y - z - 1 = 0 \\
                                2x - y - 2z + 3  = 0
                        \end{dcases}
                $.
        }

        La generica equazione di un piano \`{e}
        \[
                ax_0 + by_0 + cz_0 + d = 0
        \]
        ovvero \`{e} descritta mediante un punto interno al piano, un vettore ortogonale al piano ed una costante. Conoscendo un punto interno al piano ed un vettore ortogonale al piano \`{e} possibile ricavare la costante $d$.

        \paragraph{}
        Le coordinate del punto interno sono note ed il vettore ortogonale al piano \`{e} dato dal vettore direzione della retta $t$ perpendicolare al piano.

        \paragraph{}
        Essendo per\`{o} $t$ descritta mediante due equazioni del piano, per ricavare il vettore direzione di $t$ bisogna eseguire il prodotto tra i vettori dei dua piani che la descrivono, ottenendo cos\`{i} il vettore ortogonale, ovvero il vettore direzione di $t$.

        \paragraph{}
        $(a_1, b_1, c_1)$ e $(a_2, b_2, c_2)$ sono i due vettori ortogonali ai piani che descrivono la retta $t$ e sono dati rispettivamente dai coefficienti dei due piani.

        \paragraph{}
        Quindi
        \[
                \begin{aligned}
                        v_t & = (a_1, b_1, c_1) \times (a_2, b_2, c_2) = det\begin{pmatrix}
                                x & y & z \\
                                a_1 & b_1 & c_1 \\
                                a_2 & b_2 & c_2
                        \end{pmatrix} = [ (b_1 c_2 - c_1 b_2), (c_1 a_2 - a_1 c_2), (a_1 b_2 - b_1 a_2)] \\
                        & = (-5, 0, -5).
                \end{aligned}
        \]

        \paragraph{}
        $-5 \cdot 1 + 0 \cdot 1 - 5 \cdot 1 + d = 0 \implies d = 10.$

        \paragraph{}
        \framebox{$-5x -5z + 10 = 0.$}

        \newpage


        %%%%%%%%%%%%%%%%%%%%%%
        \section*{Esercizio 2}
        %%%%%%%%%%%%%%%%%%%%%%
        \textsl{
                Determinare l'equazione della retta passante per $P(3,2)$ tangente la circonferenza
                \[ x^2 + y^2 - 2x -2y = 1. \]
        }

        Per prima cosa si verifica se il punto appartiene alla circonferenza.

        \[
                9 + 4 -6 -4 -1 = 0 \implies 2 = 0. \mbox{ $P$ non appartiene alla circonferenza. }
        \]

        Il fascio di rette passanti per il punto P è
        \[
                y - 2 = m(x - 3);
        \]
        \[
                mx -y -3m + 2 = 0.
        \]

        Le soluzioni del sistema dato dall'equazione della circonferenza e quella della retta
        sono i punti di intersezione. Per risolvere il sistema e trovare la tangente si impone
        la condizione di tangenza, ovvero $\Delta = b^2 - 4ac = 0$ sul polinomio di secondo grado.
        \[
                \begin{dcases}
                        x^2 + y^2 - 2x -2y - 1 = 0 \\
                        mx -y -3m + 2 = 0
                \end{dcases} \quad \implies \quad
                \begin{aligned}
                        x^2 + (mx -3m + 2)^2 - 2x -2(mx -3m + 2) - 1 = 0 \\
                        y = mx -3m + 2
                \end{aligned}
        \]
        \[
                \quad \implies \quad
                \begin{aligned}
                        & x^2 + m^2x^2 - 3m^2x + 2mx - 3m^2x 9m^2 - 6m + 2mx \mbox{ \cancel{- 6m}} \mbox{ \cancel{+ 4}} - 2x -2mx \mbox{ \cancel{+ 6m}} \mbox{ \cancel{-4}} - 1 = 0 \\
                        & y = mx -3m + 2
                \end{aligned}
        \]

        TODO

        \newpage


        %%%%%%%%%%%%%%%%%%%%%%
        \section*{Esercizio 3}
        %%%%%%%%%%%%%%%%%%%%%%
        \textsl{
                Si cosideri l'equazione $x^2 + y^2 + kx -3ky - 1 = 0$.
        }

        \paragraph{}
        \textsl{
                a) Dire per quali valori di $k \in \mathbb{R}$ rappresenta una circonferenza.
        }

        Per qualsiasi $k$.

        \paragraph{}
        \textsl{
                b) Determinare i valori di $k$ per cui la circonferenza interseca gli assi cartesiani nei punti $P_1,P_2$ tali che $d(P_1,P_2) = 3$.
        }

        \paragraph{}
        Per determinare le intersezioni con gli assi è sufficiente mettere a sistema l'equzione della circonferenza con le equazioni degli assi cartesiani, ed imporre la distanza fra due punti uguale a 3.

        \[
                P(0,\ldots) = \begin{dcases}
                        x^2 + y^2 + kx -3ky - 1 = 0 \\
                        x = 0
                \end{dcases}
                \quad ; \quad
                \begin{aligned}
                        y^2 -3ky - 1 = 0 \\
                        x = 0
                \end{aligned}
                \quad ; \quad
                \begin{aligned}
                        y_{1,2} = 3k \pm \sqrt{9k^2 + 4} \\
                        x = 0
                \end{aligned}\;.
        \]

        \[
                P(\ldots,0) = \begin{dcases}
                        x^2 + y^2 + kx -3ky - 1 = 0 \\
                        y = 0
                \end{dcases}
                \quad ; \quad
                \begin{aligned}
                        x^2 + kx - 1 = 0 \\
                        y = 0
                \end{aligned}
                \quad ; \quad
                \begin{aligned}
                        x = -k \pm \sqrt{k^2 + 4} \\
                        y = 0
                \end{aligned}\;.
        \]

\end{document}
